\documentclass[aps,prd,reprint,twocolumn,groupedaddress]{revtex4-1}

\usepackage{amssymb,amsmath}
\usepackage{epsfig,hyperref}
\usepackage[usenames,svgnames]{xcolor}

\hypersetup{
    colorlinks=true,
    urlcolor=SteelBlue,
    linkcolor=red,
    citecolor=blue,
}

\renewcommand{\vec}[1]{\mathbf{#1}}

\begin{document}
%\preprint{}

\title{Scalar field accretion in Schwarzschild-like spacetimes}

\author{Andrei~V.~Frolov}
\email{frolov@sfu.ca}
\affiliation{Physics Department, Simon Fraser University\\
8888 University Drive, Burnaby, BC, Canada, V5A~1S6}

\date{\today}

\begin{abstract}
  This note describes a tutorial code solving radial scalar field wave equation in a Schwarzschild-like spacetime. As we are looking for smooth solutions, the method of choice to calculate spatial derivatives is pseudo-spectral, using Chebyshev basis on compactified coordinate. Absorbing boundary conditions are implemented using perfectly matched layers, applied to flux-conservative form of the wave equation. Time integration is done using implicit Gauss-Legendre method, which is A-stable and symplectic for Hamiltonian problems. Spherical collapse in scalar-tensor theories usually reduces to this type of PDE's. Please cite \texttt{arXiv:XXXX.XXXXX} if you use this code for your research.
\end{abstract}

%\pacs{}
%\keywords{}

\maketitle

\section{Scalar Field Equations of Motion}

Equations of motion describing evolution of a scalar field $\phi$ with a (non-linear) self-interaction potential $V(\phi)$ and an external force term ${\cal F}$ propagating on a fixed background spacetime are described by a (semi-linear) PDE
\begin{equation}
  \Box\phi = V'(\phi) - {\cal F},
\end{equation}
where $\Box$ denotes a covariant d'Alembert operator. For a spherically symmetric black hole described by the Schwarzschild metric
\begin{equation}
  ds^2 = -g(r)\, dt^2 + \frac{dr^2}{g(r)} + r^2\, d\Omega^2,
\end{equation}
where $d\Omega^2$ is the metric on a unit sphere, and the metric function $g(r)$ is
\begin{equation}
  g(r) = 1 - \frac{2M}{r},
\end{equation}
the left hand side of the equation of motion is simply
\begin{eqnarray}
  \Box\phi &=& 
  \frac{1}{\sqrt{-g}}\, \partial_\mu \Big( \sqrt{-g}\, g^{\mu\nu}\, \partial_\nu \phi \Big)\nonumber\\ &=&
  - \frac{1}{g(r)}\, \partial_t^2 \phi + \frac{1}{r^2}\, \partial_r \Big( r^2 g(r)\, \partial_r \phi \Big).
\end{eqnarray}
This can be reduced to a one-dimensional wave equation with constant propagation speed by introducing the tortoise coordinate $x$ by $\partial_x = g(r)\, \partial_r$. With this re-definition, the scalar field equation of motion reads
\begin{equation}\label{eq:wave}
  - \partial_t^2 \phi + \frac{1}{r^2}\, \partial_x \Big( r^2\, \partial_x \phi \Big) = g \Big( V'(\phi) - {\cal F} \Big).
\end{equation}
Explicit form of the tortoise coordinate $x$ for the Schwarzschild spacetime can be obtained by integrating
\begin{equation}
  x = \int \frac{dr}{g(r)} = r + 2M\,\ln\left(\frac{r}{2M} - 1\right).
\end{equation}
Tortoise coordinate $x$ is vastly preferable for numerical integration of the wave equation over areal coordinate $r$ since the characteristic speed is constant on the sampled time slice, but the added difficulty with this choice is that accurate $r(x)$ inversion is quite not-trivial numerically, as detailed in Appendix~\ref{sec:tortoise}.

The standard numerical evolution scheme would involve first-order Hamiltonian dynamical system
\begin{equation}
  \dot\phi = \pi, \hspace{1em}
  \dot\pi = \frac{1}{r^2}\, \partial_x \Big( r^2\, \partial_x \phi \Big) - g \Big( V'(\phi) - {\cal F} \Big).
\end{equation}
However, as we will see in the following Section, it is easier to handle absorbing boundary conditions if we rewrite equations of motion in a flux-conservative form by introducing auxiliary variables $u \equiv \partial_t \phi$ and $v \equiv r^2 \partial_x \phi$, so that equations of motion become
\begin{eqnarray}\label{eq:flux}
  - \partial_t u + \frac{1}{r^2}\, \partial_x v &=& g \Big( V'(\phi) - {\cal F} \Big),\nonumber\\
  - \partial_t v + r^2\, \partial_x u &=& 0.
\end{eqnarray}
The first equation is the identical rewrite of the wave equation (\ref{eq:wave}), while the second is the integrability condition requiring that the partial derivatives of $\phi$ commute.

\section{Absorbing Boundary Conditions}

The scalar field degree of freedom $\phi$ asymptotes to a free field evolution near horizon (where $g \rightarrow 0$), and a massive field evolution far away from the black hole (where $V'(\phi) \rightarrow {\cal F}$). Physically, excitations in $\phi$ take infinite amount of  time $t$ to reach both boundaries, yet truncating or compactifying the evolution domain for numerical purposes will inevitably lead to spurious reflections unless special care is taken. The best technique to avoid spurious reflections is to introduce absorbing boundary conditions via Perfectly Matched Layers (PMLs), which damp the solution at  the boundaries while guaranteeing identically vanishing reflection coefficient at the absorption layer \cite{Johnson:wt}. This is achieved by analytic continuation of the equations of motion into the complex domain
\begin{equation}
  x \rightarrow x + i f(x), \hspace{1em}
  \partial_x \rightarrow \frac{\partial_x}{1 + i f'(x)} \equiv
  \frac{\partial_x}{1 + \frac{\gamma(x)}{\partial_t}},
\end{equation}
which turns the oscillatory travelling waves $e^{ikx-i\omega t}$ into exponentially decaying functions of $x$ instead. To make attenuation length independent of $\omega$, frequency dependent contour deformation $f' = \gamma(x)/\omega$ is chosen and $i/\omega$ is transformed back into explicit integration operator in the time domain. Applying this idea to the scalar field equations of motion in flux-conservative form (\ref{eq:flux}) for an arbitrary damping function $\gamma(x)$, we obtain
\begin{eqnarray}
  - (\partial_t + \gamma) u + \frac{1}{r^2}\, \partial_x v &=& \left(1 + \frac{\gamma(x)}{\partial_t}\right) \Bigg[g \Big( V'(\phi) - {\cal F} \Big)\Bigg],\nonumber\\
  - (\partial_t + \gamma) v + r^2\, \partial_x u &=& 0.
\end{eqnarray}
To turn the inverse time evolution operator $\partial_t^{-1}$ into a differential equation form, introduction of a third auxiliary variable $w$ is in order. With re-definition $u \rightarrow u+w$, the non-reflecting PML equations of motion then become
\begin{subequations}\label{eq:pml}
\begin{eqnarray}
  \partial_t \phi &=& u-w,\\
  \partial_t u &=& \frac{1}{r^2}\, \partial_x v - \gamma u,\\
  \partial_t v &=& r^2\, \partial_x (u-w) - \gamma v,\\
  \partial_t w &=& g \Big( V'(\phi) - {\cal F} \Big).
\end{eqnarray}
\end{subequations}
The damping function $\gamma(x)$ can be quite arbitrary, but it should have compact support near the boundaries to not affect the evolution in the interior, and have sufficient support and magnitude to absorb the impinging waves which hit the boundary during the expected evolution.

\section{Spectral Basis}

As the scalar field is usually quite stiff and does not form shocks in the course of evolution, the method of choice to evaluate derivative operators is spectral \cite{Boyd}. Compactifying the tortoise coordinate $x$ on a scale $\ell$
\begin{equation}
  y = \frac{x}{\sqrt{x^2+\ell^2}} \equiv \cos\theta, \hspace{1em}
  \frac{x}{\ell} = \frac{y}{1-y^2} = \cot\theta
\end{equation}
and introducing a Chebyshev basis on interval $y \in [-1,1]$
\begin{subequations}
\begin{eqnarray}
  T_n &=& \cos(n\theta),\\
  \partial_x T_n &=& \frac{n}{\ell}\, \sin(n\theta)\, \sin^2\theta,\\
  \partial_x^2 T_n &=& \frac{n}{\ell^2} \Big(n \cos(n\theta) + 2 \cot\theta \sin(n\theta)\Big) \sin^4\theta,
\end{eqnarray}
\end{subequations}
we arrive at the spectral representation of the solution
\begin{equation}
  \phi(x) = \sum\limits_n c_n T_n(y)
\end{equation}
truncated to a finite number of modes. While Galerkin method to discretize equations of motion can be employed, the simplest method to evaluate derivative operators is pseudo-spectral, where equations of motion are solved on a Gauss-Lobatto grid
\begin{equation}
  \theta_i = \left(n-i+\frac{1}{2}\right)\,\frac{\pi}{n}, \hspace{1em}
  x_i = \ell\cot\theta_i.
\end{equation}
One does not have to explicitly find coefficients $c_n$ to evaluate the derivative operators of a function $\phi(x)$ sampled on a collocation grid $x_i$. Instead, derivative operators like ${\cal D}_{ij}$ and ${\cal L}_{ij}$ can be found in advance by solving linear matrix equations
\begin{subequations}
\begin{eqnarray}
  \sum\limits_j {\cal D}_{ij} T_n(x_j) &=& \partial_x T_n(x_i),\\
  \sum\limits_j {\cal L}_{ij} T_n(x_j) &=& \left(\partial_{x}  + \frac{2g}{r}\right) \partial_x T_n(x_i),
\end{eqnarray}
\end{subequations}
and so on for every basis function $T_n$ evaluated at all nodes $x_i$.

\section{Gauss-Legendre Integrator}

Packing the scalar field variables $\phi,u,v,w$ evaluated at the collocation grid points $x_i$ into a state vector $\vec{y} \equiv \{ \phi(x_i), u(x_i), v(x_i), w(x_i) \}$, the wave equation (\ref{eq:pml}) reduces to an autonomous dynamical system
\begin{equation}
  \frac{d\vec{y}}{dt} = \vec{f}(\vec{y}),
\end{equation}
which can be integrated by an implicit Runge-Kutta method \cite{Butcher:1964fx}
\begin{equation}
  \vec{y} \rightarrow \vec{y} + \Delta t \cdot \sum\limits_i b_i \vec{g}^{(i)},
\end{equation}
where the trial directions $\vec{g}^{(i)}$ are defined by
\begin{equation}
  \vec{g}^{(i)} = \vec{f}\left(\vec{y} + \Delta t \cdot \sum\limits_j a^i_j \vec{g}^{(j)}\right).
\end{equation}
Particularly accurate choice of coefficients for a time integrator corresponds to a Gauss-Legendre quadrature, where the trial directions are evaluated at the zeroes of the (shifted) Legendre polynomial
\begin{equation}
  P_n\left(2c^{(i)}-1\right) = 0,
\end{equation}
with coefficients $a^i_j$ and $b_j$ set by
\begin{eqnarray}
  \sum\limits_j a^i_j \left[c^{(j)}\right]^{k-1} &=& \frac{1}{k}\, \left[c^{(i)}\right]^k\\
  \sum\limits_j b_j \left[c^{(j)}\right]^{k-1} &=& \frac{1}{k}.
\end{eqnarray}
The resulting time integration method is A-stable and symplectic for Hamiltonian problems, and is extremely easy to implement using a simple iterative scheme.


\section{Static Solver}

Static configurations of the field $\phi$ have $\partial_t \phi = 0$ and can be found by solving a (semi-linear) elliptical problem
\begin{equation}
  {\cal L}\phi = g \Big( V'(\phi) - {\cal F} \Big).
\end{equation}
One can improve a trial solution $\bar\phi$ using Newton's method by linearizing $\phi = \bar\phi + \delta\phi$ and solving
\begin{equation}
  {\cal L}(\bar\phi + \delta\phi) = g \Big( V'(\bar\phi + \delta\phi) - {\cal F} \Big),
\end{equation}
which translates the residual ${\cal R} = -{\cal L}\bar\phi + g \Big( V'(\bar\phi) - {\cal F} \Big)$ into a correction $\delta\phi$ by solving a set of linear equations
\begin{equation}
  \Big({\cal L} - g V''(\bar\phi)\Big) \delta\phi = -{\cal L}\bar\phi + g \Big( V'(\bar\phi) - {\cal F} \Big).
\end{equation}
With the basis as chosen in the last Section, this scheme converges to machine precision in about 16 iterations or so for most potentials.

\appendix
\section{Inverting Tortoise Coordinate}
\label{sec:tortoise}

Accurately inverting Schwarzschild tortoise coordinate
\begin{equation}
  x = r + 2M\,\ln\left(\frac{r}{2M} - 1\right)
\end{equation}
to yield areal coordinate $r$ as a function of $x$ turns out to be a rather non-trivial task, despite appearances. The problem is that asymptotic for large positive $x$, where $r \simeq x - 2M \ln\left(x/2M - 1\right)$, and for large negative $x$, where $r \simeq 2M$ with exponentially suppressed metric function $\ln g \simeq x/2M - 1$, have vastly different derivatives with respect to $x$ (which hampers numerical schemes like Newton's method), and no closed form algebraic inverse.

%\begin{equation}
%  \delta r = -\Bigg(r + 2M\,\ln\left(\frac{r}{2M} - 1\right) - x\Bigg) \cdot \frac{dr}{dx}, \hspace{1em}
%  \frac{dr}{dx} = g(r)
%\end{equation}
%\begin{equation}
%  \delta \ln g = -\Bigg(\frac{2M}{1-g} + 2M \ln \frac{g}{1-g} - x\Bigg) \cdot \frac{d\ln g}{dx}, \hspace{1em}
%  \frac{d\ln g}{dx} = \frac{(1-g)^2}{2M}
%\end{equation}

A trick that works for the entire usable range of $x$ is to solve for an approximation variable $q \simeq x - 2M$ instead \begin{equation}
  q = 2M\, \ln\Bigg(\exp\left(\frac{r}{2M} - 1\right) - 1\Bigg),
\end{equation}
which (unlike $x$) is easily invertible in terms of $r$
\begin{equation}
  r = 2M \Bigg(1 + \ln\left(1+\exp\frac{q}{2M}\right)\Bigg),
\end{equation}
and can be readily found by Newton's method iterating $q \rightarrow q + \delta q$ with
\begin{equation}
  \delta q = -\Bigg(r + 2M\,\ln\left(\frac{r}{2M} - 1\right) - x\Bigg) \cdot \frac{dq}{dx},
\end{equation}
as the derivative
\begin{equation}
  \frac{dq}{dx} = \left(1+\exp\frac{-q}{2M}\right) g(r)
\end{equation}
is of order one on the entire domain of definition of $x$. One still has to be careful to avoid numerical overflows in the exponents or catastrophic loss of precision when taking logarithms of one plus a small number, which can be achieved by evaluating 
\begin{equation}
  \ln\left(1+e^q\right) = \left\{ \begin{array}{rr}
    q + \ln\left(1 + e^{-q}\right),& q \ge 0\\
    2\, \text{atanh}\, {\displaystyle\frac{e^q}{2+e^q}},& q < 0\\
  \end{array} \right.
\end{equation}
in different limits.

\bibliography{accretion}

\end{document}
